\documentclass[]{scrartcl}
\usepackage{graphicx}
\usepackage{amsmath} % For equation alignment, bmatrix
\usepackage{amssymb} % For special math characters such as E with two vertical lines (expectation value symbol), Real numbers symbol, etc
\usepackage{witharrows} % For an aligned math environment where arrows show which transformation steps were taken
%\usepackage{mathtools} % For some extra math-related functionality, such as matrix*
%\usepackage{hyperref} % For using \autoref
%\usepackage{listings} % For adding code to the latex document
%\usepackage{caption} % For adding captions
\usepackage{color} % For color in code
%\usepackage{nicematrix} % For creating matrices with outer rows and columns, with dsahed line separators. Documentation: https://ctan.org/pkg/nicematrix
\usepackage{float} % For better control over float environments
%\usepackage[utf8]{inputenc} % this is needed for umlauts
%\usepackage[ngerman]{babel} % this is needed for umlauts
%\usepackage[T1]{fontenc}    % this is needed for correct output of umlauts in pdf




% Opening / Title
\title{Complex Systems in Bioinformatics \\ \vspace{2mm} Exercise 6 \\ \vspace{2mm}}
\subtitle{Lecturers: Prof. Dr. Max von Kleist, Prof. Dr. Jana Wolf, Prof. Dr. Martin Vingron}
\author{Kristian Reinhart, 4474140 \\ Duong Ha Le Minh, 5314209}
\newenvironment{tightcenter}{%
  \setlength\topsep{0pt}
  \setlength\parskip{0pt}
  \begin{center}
}{%
  \end{center}
}




%%%%%%%%%%%%%%%%%%%%%%%%%%%%%%%%%%%
%%%	Begin actual document	%%%
%%%%%%%%%%%%%%%%%%%%%%%%%%%%%%%%%%%


\begin{document}




\maketitle




\section*{6. Exercise (Block II-Assignment 2)}

%%%%%%%%%%%%%%%%%%%%%%%%%%%%%%%%%%%
%%%			Exercise ODE		%%%
%%%%%%%%%%%%%%%%%%%%%%%%%%%%%%%%%%%

\subsection*{ODE System}

\textit{Consider the system of differential equations:
\begin{center}
% \frac{dx}{dt} & = \dot{x} =  v_0 - k_1 x y^2 & \hspace{4em} & \rlap{\footnotesize (Right-side comment if wanted)}  \\
\begin{align*}
	\frac{dx}{dt} & = \dot{x} =  v_0 - k_1 x y^2 \\
	\frac{dy}{dt} & = \dot{y} =  k_1 x y^2 - k_2 y \\
\end{align*}
\end{center}
}



%%%%%%%%%%%%%%%%%%%
%%%	Task a)	%%%
%%%%%%%%%%%%%%%%%%%

\subsubsection*{a)}

To determine steady states of the system all equations of the ODE must equate to 0.
\\
As the highest-order term of the equations is of the second order we can expect up to two steady states for the given system.
\\
We begin with the second equation
\begin{center}
\[
\begin{WithArrows}
	0 & = k_1 x y^2 - k_2y \Arrow{ factor out y} \\
	0 & = y(k_1 x y - k_2) \Arrow{$|:y,~ |+k_2,~ |:k_1 y$} \\
	\frac{k_2}{k_1 y} & = x
\end{WithArrows}
\]
\end{center}

\noindent We can now insert the determined value for $x$ into the first equation:
\begin{center}
\[
\begin{WithArrows}
	0 & = v_0 - k_1 \frac{k_2}{k_1 y} y^2 \Arrow{cancel $k_1$ and $y$} \\
	0 & = v_0 - k_2 y \Arrow{$|-v_0,~ |:k_2,~|*(-1)$} \\
	\frac{v_0}{k_2} & = y
\end{WithArrows}
\]
\end{center}

\noindent As the $x$ we got from the first equation is still dependent on $y$ we can replace it now:
\begin{center}
\[
\begin{WithArrows}
	x & = \frac{k_2}{k_1 \left( \frac{v_0}{k_2} \right)} \Arrow{Rewrite double fraction} \\
	x & = \frac{k_2^2}{•k_1 v_0}
\end{WithArrows}
\]
\end{center}

\noindent Therefore we obtain the only steady state of the system at $\left( \frac{k_2^2}{•k_1 v_0} , \frac{v_0}{k_2} \right)$.

%%%%%%%%%%%%%%%%%%%
%%%	Task b)	%%%
%%%%%%%%%%%%%%%%%%%

\subsubsection*{b)}

\noindent We first rewrite our ODE as a vector of functions:

\begin{center}
\[
f(x,y) ~=~ \left[ \begin{array}{c} f_1(x,y) \\ f_2(x,y) \end{array} \right] ~=~ \left[ \begin{array}{c} v_0 - k_1 x y^2 \\ k_1 x y^2 - k_2 y \end{array} \right]
\]
\end{center}

\noindent The Jacobi matrix for our 2x2 system is defined as

\begin{center}
\[
J(x,y) ~=~ \frac{\partial f}{\partial x}(a) ~=~
\begin{bmatrix}
  \frac{\partial f_1}{\partial x} & \frac{\partial f_1}{\partial y} \\[1ex] % <-- 1ex more space between rows of matrix
  \frac{\partial f_2}{\partial x} & \frac{\partial f_2}{\partial y}
\end{bmatrix}
\]
\end{center}

\noindent With the Jacobi matrix general form we can now calculate the partial derivatives:

\begin{center}
\begin{align*}
	\frac{\partial f_1}{\partial x} &= \frac{\partial}{\partial x}(v_0 - k_1 x y^2) &= & -k_1 y^2 \\
	\frac{\partial f_1}{\partial y} &= \frac{\partial}{\partial y}(v_0 - k_1 x y^2) &= & -2 k_1 x y \\
	\frac{\partial f_2}{\partial x} &= \frac{\partial}{\partial x}(k_1 x y^2 - k_2 y)	&= & k_1 y^2 \\
	\frac{\partial f_2}{\partial y} &= \frac{\partial}{\partial y}(k_1 x y^2 - k_2 y)	&= & 2 k_1 x y - k_2 \\
\end{align*}
\end{center}

\noindent Having determined the partial derivatives we can write the completed Jacobi matrix:

\begin{center}
\[
J(x,y) ~=~
\begin{bmatrix}
  -k_1 y^2 & -2 k_1 x y \\[1ex] % <-- 1ex more space between rows of matrix
   k_1 y^2 & 2 k_1 x y - k_2
\end{bmatrix}
\]
\end{center}

\noindent With the completed Jacobi matrix we can no insert the steady state of the ODE $\left( \frac{k_2^2}{k_1 v_0} , \frac{v_0}{k_2} \right)$ into the Jacobi matrix:

\begin{center}
\[
J \left( \frac{k_2^2}{k_1 v_0} , \frac{v_0}{k_2} \right) ~=~
%\begin{bmatrix}
%  -k_1 \left( \frac{v_0}{k_2} \right)^2 & -2 k_1 \left( \frac{k_2^2}{•k_1 v_0} \right) \left( \frac{v_0}{k_2} \right) \\[1ex] % <-- 1ex more space between rows of matrix
%   k_1 \left( \frac{v_0}{k_2} \right)^2 & 2 k_1 \left( \frac{k_2^2}{•k_1 v_0} \right) \left( \frac{v_0}{k_2} \right) - k_2
%\end{bmatrix} ~=~
\begin{bmatrix}
  -\left( \frac{k_1 v_0^2}{k_2^2} \right) & -2 k_2 \\[1ex] % <-- 1ex more space between rows of matrix
   \left( \frac{k_1 v_0^2}{k_2^2} \right) & 2 k_2 - k_2
\end{bmatrix} ~=~
\begin{bmatrix}
  -\left( \frac{k_1 v_0^2}{k_2^2} \right) & -2 k_2 \\[1ex] % <-- 1ex more space between rows of matrix
   \left( \frac{k_1 v_0^2}{k_2^2} \right) & k_2
\end{bmatrix}
\]
\end{center}


%%%%%%%%%%%%%%%%%%%
%%%	Task c)	%%%
%%%%%%%%%%%%%%%%%%%

\subsubsection*{c)}

\noindent The trace of a matrix $A$ is defined as the sum of its diagonal entries:

\begin{center}
$
tr(A) ~=~ \sum_{i=1}^n a_{ii}
$
\end{center}

\noindent The trace for our Jacobi matrix at the steady state is

\begin{center}
$
tr \left( J \left( \frac{k_2^2}{k_1 v_0} , \frac{v_0}{k_2} \right) \right) ~=~ -\left( \frac{k_1 v_0^2}{k_2^2} \right) + k_2
$
\end{center}

\noindent The determinant for our Jacobi matrix at the steady state is

\begin{center}
\begin{align*}
det \left( J \left( \frac{k_2^2}{k_1 v_0} , \frac{v_0}{k_2} \right) \right) &= -\left( \frac{k_1 v_0^2}{k_2^2} \right) * k_2 - \left( -2 k_2 * \frac{k_1 v_0^2}{k_2^2} \right) \\
&= - \frac{k_1 v_0^2}{k_2} + \frac{2 k_1 v_0^2}{k_2} \\
&= \frac{k_1 v_0^2}{k_2}
\end{align*}
\end{center}


%%%%%%%%%%%%%%%%%%%
%%%	Task d)	%%%
%%%%%%%%%%%%%%%%%%%

\subsubsection*{d)}

\noindent We begin with defining the general form for the matrix to illustrate which values to apply to the formulas:

\begin{center}
$
	A = \begin{pmatrix}
		   a & b \\[1ex] % <-- 1ex more space between rows of matrix
		   c & d
		\end{pmatrix}
$
\end{center}

\noindent We then continue with the characteristic equation for A:
\begin{center}
\begin{align*}
	det(A - \lambda I) = \lambda^2 - & (\underbrace{a+d}) & \lambda * & (\underbrace{ad-bc}) & = 0 \\
								   	 & 			  S=tr(J) & 		  & 			P=det(J) &	   \\
\end{align*}
\end{center}

\noindent From the characteristic equation we can derive the formula to determine the Eigenvalues $\lambda_{1,2}$:

\begin{center}
\begin{align*}
	\lambda_{1,2} =  & \frac{1}{2} \left( S \pm \sqrt{S^2 - 4P} \right) \\
\end{align*}
\end{center}

\noindent We can observe the required system stability behaviors if we choose the parameters such that the eigenvalues $\lambda_1, \lambda_2$ of the system fulfill the following conditions:

\begin{center}
\begin{align*}
	\lambda_1 \neq \lambda_2, & ~ \lambda_{1,2} \in \mathbb{R}, 		& ~ \lambda_{1,2} < 0	& \hspace{1em} & \rlap{\footnotesize (Stable node)}  \\
	\lambda_1 \neq \lambda_2, & ~ \lambda_{1,2} \in \mathbb{R}, 		& ~ \lambda_{1,2} > 0	& \hspace{1em} & \rlap{\footnotesize (Unstable node)}  \\
							  &											&						&			   & \\
							  & ~ \lambda_{1,2} = \alpha \pm i \beta,	& ~ \beta \neq 0 		& 			   & \rlap{\footnotesize (General complex conjugate form)} \\
	 						  & ~ \lambda_{1,2} \in \mathbb{C}, 		& ~ \alpha < 0 			& \hspace{1em} & \rlap{\footnotesize (Stable focus, inward spiral)}  \\
	 						  & ~ \lambda_{1,2} \in \mathbb{C}, 		& ~ \alpha > 0 			& \hspace{1em} & \rlap{\footnotesize (Unstable focus, outward spiral)}  \\
\end{align*}
\end{center}

\noindent Additionally from the task we have the requirement that  $v_0, k_1, k_2 > 0$ and $k_1 = 1$. We now have the equation:

\begin{center}
\begin{align*}
	\lambda_{1,2}	& =  \frac{1}{2} \left( \left( - \left( \frac{k_1 v_0^2}{k_2^2} \right) + k_2 \right) \pm \sqrt{ \left( - \left( \frac{k_1 v_0^2}{k_2^2} \right) + k_2 \right)^2 - 4 \frac{k_1 v_0^2}{k_2} } \right) & \hspace{1em} & \rlap{\footnotesize (Set $k_1 = 1$)} \\
					& =	\frac{1}{2} \left( \left( - \left( \frac{v_0^2}{k_2^2} \right) + k_2 \right) \pm \sqrt{ \left( - \left( \frac{v_0^2}{k_2^2} \right) + k_2 \right)^2 - 4 \frac{v_0^2}{k_2} } \right) & & \\
					& =	\frac{1}{2} \left( \left( \frac{k_2^3 - v_0^2}{k_2^2} \right) \pm \sqrt{ \left( \frac{k_2^3 - v_0^2}{k_2^2}  \right)^2 - 4 \frac{v_0^2}{k_2} } \right) & & \\
\end{align*}
\end{center}

\noindent From the formula we can see that if $k_2^3 > v_0^2$ the first part of the equation remains positive, and if $k_2^3 < v_0^2$ it becomes negative.
We can also see that if $\left( \frac{k_2^3 - v_0^2}{k_2^2}  \right)^2 > 4 \frac{v_0^2}{k_2}$ the value inside the root is positive and thus real, whereas $\left( \frac{k_2^3 - v_0^2}{k_2^2}  \right)^2 < 4 \frac{v_0^2}{k_2}$ will lead to a negative root and therefore a complex result. With these inequalities as guides we try various combinations for $v_0, k_2 > 0$ and $k_1 = 1$.


\begin{center}
\begin{align*}
	k_1 = 1, v_0 = 2, k_2 = 1 & \\
	\lambda_{1,2}	& =  \frac{1}{2} \left( \left( \frac{1^3 - 2^2}{1^2} \right) \pm \sqrt{ \left( \frac{1^3 - 2^2}{1^2}  \right)^2 - 4 \frac{2^2}{1} } \right) \\
					& =  \frac{1}{2} \left( -3 \pm \sqrt{ 9 - 16 } \right) \\
\end{align*}
\end{center}

With the real part negative and the square root negative and thus complex the parameter combination $k_1 = 1, v_0 = 2, k_2 = 1$ leads to a \textbf{stable focus}.


\begin{center}
\begin{align*}
	k_1 = 1, v_0 = 3, k_2 = 1 & \\
	\lambda_{1,2}	& =  \frac{1}{2} \left( \left( \frac{1^3 -3^2}{1^2} \right) \pm \sqrt{ \left( \frac{1^3 - 3^2}{1^2}  \right)^2 - 4 \frac{3^2}{1} } \right) \\
					& =  \frac{1}{2} \left( -8 \pm \sqrt{ 64 - 36 } \right) \\
					& =  \frac{1}{2} \left( -8 \pm \sim 5.3 \right) \\
\end{align*}
\end{center}

With $\lambda_{1,2}$ real and negative the parameter combination $k_1 = 1, v_0 = 3, k_2 = 1$ leads to a \textbf{stable node}.

\begin{center}
\begin{align*}
	k_1 = 1, v_0 = 5, k_2 = 3 & \\
	\lambda_{1,2}	& =  \frac{1}{2} \left( \left( \frac{3^3 - 5^2}{3^2} \right) \pm \sqrt{ \left( \frac{3^3 - 5^2}{3^2}  \right)^2 - 4 \frac{5^2}{3} } \right) \\
					& =  \frac{1}{2} \left( \frac{2}{9} \pm \sqrt{ \frac{4}{81} - \frac{100}{3} } \right) \\
					& =  \frac{1}{2} \left( \frac{2}{9} \pm \sqrt{ -\frac{2696}{81} } \right) \\
\end{align*}
\end{center}

With the real part positive and the square root negative and thus complex the parameter combination $k_1 = 1, v_0 = 5, k_2 = 3$ leads to an \textbf{unstable focus}.

\begin{center}
\begin{align*}
	k_1 = 1, v_0 = 1, k_2 = 2 & \\
	\lambda_{1,2}	& =  \frac{1}{2} \left( \left( \frac{2^3 - 1^2}{2^2} \right) \pm \sqrt{ \left( \frac{2^3 - 1^2}{2^2}  \right)^2 - 4 \frac{1^2}{2} } \right) \\
					& =  \frac{1}{2} \left( \frac{7}{4} \pm \sqrt{ \frac{49}{16} - 2 } \right) \\
					& =  \frac{1}{2} \left( 1,75 \pm \sim 1,03 \right) \\
\end{align*}
\end{center}

With $\lambda_{1,2}$ real and positive the parameter combination $k_1 = 1, v_0 = 1, k_2 = 2$ leads to a \textbf{unstable node}.
\\
\\
In summary, for the given ODE:

\begin{center}
\begin{align*}
	k_1 = 1, ~ v_0 = 2, ~ k_2 = 1 & \hspace{1em} & \rlap{stable focus} \\
	k_1 = 1, ~ v_0 = 3, ~ k_2 = 1 & \hspace{1em} & \rlap{stable node} \\
	k_1 = 1, ~ v_0 = 5, ~ k_2 = 3 & \hspace{1em} & \rlap{unstable focus} \\
	k_1 = 1, ~ v_0 = 1, ~ k_2 = 2 & \hspace{1em} & \rlap{unstable node} \\
\end{align*}
\end{center}

%%%%%%%%%%%%%%%%%%%
%%%	Task e)	%%%
%%%%%%%%%%%%%%%%%%%

\subsubsection*{e)}

The Trace=0 separates the stable (Trace<0) from the unstable (Trace>0) regions.

%%%%%%%%%%%%%%%%%%%
%%%	Task f)	%%%
%%%%%%%%%%%%%%%%%%%

\subsubsection*{f)}






\end{document}



