\documentclass[]{scrartcl}
%\usepackage{graphicx}
%\usepackage{amsmath} % For equation alignment
\usepackage{mathtools} % For some extra math-related functionality, such as matrix*
%\usepackage{hyperref} % For using \autoref
%\usepackage{listings} % For adding code to the latex document
%\usepackage{caption} % For adding captions
\usepackage{color} % For color in code
\usepackage{nicematrix} % For creating matrices with outer rows and columns, with dsahed line separators. Documentation: https://ctan.org/pkg/nicematrix
%\usepackage{float} % For better control over float environments
%\usepackage[utf8]{inputenc} % this is needed for umlauts
%\usepackage[ngerman]{babel} % this is needed for umlauts
%\usepackage[T1]{fontenc}    % this is needed for correct output of umlauts in pdf




% Opening / Title
\title{Complex Systems in Bioinformatics \\ \vspace{2mm} Block 3 Project 2 \\ \vspace{2mm}}
\subtitle{Lecturers: Prof. Dr. Max von Kleist, Prof. Dr. Jana Wolf, Prof. Dr. Martin Vingron}
\author{Duong Ha Le Minh, 5314209 \\ Kristian Reinhart, 4474140}
\newenvironment{tightcenter}{%
  \setlength\topsep{0pt}
  \setlength\parskip{0pt}
  \begin{center}
}{%
  \end{center}
}




%%%%%%%%%%%%%%%%%%%%%%%%%%%%%%%%%%%
%%%	Begin actual document	%%%
%%%%%%%%%%%%%%%%%%%%%%%%%%%%%%%%%%%


\begin{document}




\maketitle



\section*{Block III, Project 2}

\subsection*{Model System}


\begin{center}
\noindent \begin{minipage}{.4\linewidth}
\begin{flalign*}
	R_1: ~ & z_1 		\overset{r_0}{\rightarrow} z_1 + z_1 &&\\
	R_2: ~ & z_1 + z_2	\overset{r_2}{\rightarrow} z_2 &&\\
	R_3: ~ & z_1 + z_2 	\overset{r_3}{\rightarrow} z_2 + z_2 &&\\
	R_4: ~ & z_2 		\overset{r_4}{\rightarrow} \emptyset &&
\end{flalign*}
\end{minipage}
\noindent \begin{minipage}{.4\linewidth}
\begin{flalign*}
	r_1: ~ & k_1 * z_1 &&\\
	r_2: ~ & k_2 * z_1 * z_2 &&\\
	r_3: ~ & k_3 * z_1 * z_2 &&\\
	r_4: ~ & k_4 * z_2 &&
\end{flalign*}
\end{minipage}
\end{center}



\subsection*{Stoichiometric Matrix and Reaction Rate Vector}

% Have the matrix in a float-environment and center it, looks nicer
\begin{center}
\noindent \begin{minipage}{.4\linewidth}
$
S =
\begin{pNiceMatrix}[first-row,last-col,nullify-dots]
	R_1	&	R_2 &	R_3 &	R_4 &		\\
	  1 &	 -1 &	 -1 &	  0 &	z_1 \\
	  0 &	  0 &	  1 &	 -1 &	z_2 \\
\end{pNiceMatrix}
$
\end{minipage}
\noindent \begin{minipage}{.4\linewidth}
$
R =
\begin{pNiceMatrix}[last-col,nullify-dots]
	k_1 * z_1		& r_1 \\
	k_2 * z_1 * z_2 & r_2 \\
	k_3 * z_1 * z_2 & r_3 \\
	k_4 * z_2		& r_4 \\
\end{pNiceMatrix}
$
\end{minipage}
\end{center}



\subsection*{ODE}

\begin{center}
\noindent \begin{minipage}{.5\linewidth}
\begin{align*}
	\frac{dz_1}{dt} & = k_1 z_1 - k_2 z_1 z_2 - k_3 z_1 z_2 & =  k_1 z_1 - (k_2 + k_3) z_1 z_2 \\
	\frac{dz_2}{dt} & = k_3 z_1 z_2 - k_4 z_2 & \\
\end{align*}
\end{minipage}
\end{center}


\begin{center}
\noindent \begin{minipage}{.5\linewidth}
\begin{align*}
	\frac{dz_1}{dt} & = k_1 z_1 - (k_2 + k_3) z_1 z_2 \\
	\frac{dz_2}{dt} & = - k_4 z_2 + k_3 z_1 z_2 & \\
\end{align*}
\end{minipage}
\end{center}


\subsection*{Nullclines}

\begin{center}
\noindent \begin{minipage}{.45\linewidth}
\begin{align*}
	\frac{dz_1}{dt} & = 0 \\
	0 & = k_1 z_1 - (k_2 + k_3) z_1 z_2 \\
	0 & = z_1 (k_1 - (k_2 + k_3) z_2) \\
	- k_1 & = - (k_2 + k_3) z_2 \\
	z_2 & = \frac{k_1}{k_2 + k_3} \\
	\textrm{Nullcline}: z_1 = 0 & ~ \textrm{or} ~ z_2 = \frac{k_1}{k_2 + k_3} = \frac{0.3}{0.01 + 0.01} = 15 \\
\end{align*}
\end{minipage}
\noindent \begin{minipage}{.45\linewidth}
\begin{align*}
	\frac{dz_2}{dt} & = 0 \\
	0 & = - k_4 z_2 + k_3 z_1 z_2 \\
	0 & = z_2 (- k_4 + k_3 z_1) \\
	z_1 & = \frac{k_4}{k_3} \\
	\textrm{Nullcline}: z_2 = 0 & ~ \textrm{or} ~ z_1 = \frac{k_4}{k_3} = \frac{0.3}{0.01} = 30 \\
\end{align*}
\end{minipage}
\end{center}

\subsection*{Fixed Points}

\begin{center}
\noindent \begin{minipage}{.45\linewidth}
\begin{align*}
	\textrm{Fixed points:} ~ & (0,0), & \left( \frac{k_4}{k_3} , \frac{k_1}{k_2 + k_3} \right) \\
\end{align*}
\end{minipage}
\end{center}


\subsection*{Stability Analysis}

\begin{center}
\noindent \begin{minipage}{.5\linewidth}
\begin{align*}
	\frac{dz_1}{dt} &= f(z_1,z_2) &= k_1 z_1 - (k_2 + k_3) z_1 z_2 \\
	\frac{dz_2}{dt} &= g(z_1,z_2) &= - k_4 z_2 + k_3 z_1 z_2 & \\
\end{align*}
\end{minipage}
\end{center}


For the given ODE we have the following Jacobi matrix:

\[
J(z_1,z_2) ~=~
\begin{bmatrix}
  \frac{\partial f}{\partial z_1} & \frac{\partial f}{\partial z_2} \\[1ex] % <-- 1ex more space between rows of matrix
  \frac{\partial g}{\partial z_1} & \frac{\partial g}{\partial z_2}
\end{bmatrix}
\]

With the Jacobi matrix general form we can now calculate the partial derivatives:

\begin{center}
\begin{align*}
	\frac{\partial f}{\partial z_1} &= k_1 - (k_2 + k_3) z_2 = 0.3 - 0.02 z_2 \\
	\frac{\partial f}{\partial z_2} &= - (k_2 + k_3) z_1 = -0.02 z_1 \\
	\frac{\partial g}{\partial z_1} &= k_3 z_2 = 0.01 z_2 \\
	\frac{\partial g}{\partial z_2} &= k_3 z_1 - k_4 = 0.01 z_1 - 0.3 \\
\end{align*}
\end{center}

We can now solve the Jacobi matrix for the fixed points:

\[
J(0,0) ~=~
\begin{bmatrix}
  k_1 & 0 \\[1ex] % <-- 1ex more space between rows of matrix
  0 & -k_4
\end{bmatrix} ~=~
\begin{bmatrix}
  0.3 & 0 \\[1ex] % <-- 1ex more space between rows of matrix
  0 & -0.3
\end{bmatrix}
\]

\begin{center}
\begin{align*}
	(0.3 -\lambda)(-0.3 -\lambda) &= 0 \\
	\lambda_1 = 0.3 & \lambda_1 = -0.3 \\
\end{align*}
\end{center}


With the real, mixed eigenvalues $\lambda_1 = 0.3$, $\lambda_2 = -0.3$ the fixed point at $(0,0)$ is an unstable saddle point.


\begin{center}
\begin{align*}
J(30,15) &=
\begin{bmatrix}
  k_1 - (k_2 + k_3) z_2 & - (k_2 + k_3) z_1 \\[1ex] % <-- 1ex more space between rows of matrix
  k_3 z_2 & k_3 z_1 - k_4
\end{bmatrix} \\
&= \begin{bmatrix}
  0.3 - (0.02) * 15 & -0.02 * 15 \\[1ex] % <-- 1ex more space between rows of matrix
  0.01 * 15 & 0.01 * 30 - 0.03
\end{bmatrix} ~=~
\begin{bmatrix}
  0 & -0.6 \\[1ex] % <-- 1ex more space between rows of matrix
  0.15 & 0
\end{bmatrix}
\end{align*}
\end{center}

\begin{center}
\begin{align*}
	\textrm{det}(J - \lambda I) &= 0 \\
	\textrm{det} \left( \left[ \begin{bmatrix}
  		0 - \lambda & -0.6 \\[1ex] % <-- 1ex more space between rows of matrix
  		0.15 & 0 - \lambda
	\end{bmatrix} \right] \right) &= 0 \\
	(-\lambda)(-\lambda) - (-0.6)(0.15) &= 0 \\
	\lambda^2 + 0.09 &= 0 \\
	\lambda_{1/2} = \pm \sqrt{-0.09} &= \pm 0.3i \\
\end{align*}
\end{center}


With the imaginary eigenvalues $\lambda_1 = +0.3i$, $\lambda_2 = -0.3i$ the fixed point at $(30,15)$ is a stable center point.



\end{document}
