\documentclass[]{scrartcl}
\usepackage{graphicx}
\usepackage{amsmath} % For equation alignment, bmatrix
\usepackage{amssymb} % For special math characters such as E with two vertical lines (expectation value symbol), Real numbers symbol, etc
\usepackage{witharrows} % For an aligned math environment where arrows show which transformation steps were taken
%\usepackage{mathtools} % For some extra math-related functionality, such as matrix*
%\usepackage{hyperref} % For using \autoref
%\usepackage{listings} % For adding code to the latex document
%\usepackage{caption} % For adding captions
\usepackage{color} % For color in code
%\usepackage{nicematrix} % For creating matrices with outer rows and columns, with dsahed line separators. Documentation: https://ctan.org/pkg/nicematrix
\usepackage{float} % For better control over float environments
%\usepackage[utf8]{inputenc} % this is needed for umlauts
%\usepackage[ngerman]{babel} % this is needed for umlauts
%\usepackage[T1]{fontenc}    % this is needed for correct output of umlauts in pdf




% Opening / Title
\title{Complex Systems in Bioinformatics \\ \vspace{2mm} Exercise 7 \\ \vspace{2mm}}
\subtitle{Lecturers: Prof. Dr. Max von Kleist, Prof. Dr. Jana Wolf, Prof. Dr. Martin Vingron}
\author{Kristian Reinhart, 4474140 \\ Duong Ha Le Minh, 5314209}
\newenvironment{tightcenter}{%
  \setlength\topsep{0pt}
  \setlength\parskip{0pt}
  \begin{center}
}{%
  \end{center}
}




%%%%%%%%%%%%%%%%%%%%%%%%%%%%%%%%%%%
%%%	Begin actual document	%%%
%%%%%%%%%%%%%%%%%%%%%%%%%%%%%%%%%%%


\begin{document}




\maketitle




\section*{7. Assignment}

%%%%%%%%%%%%%%%%%%%%%%%%%%%%%%%%%%%
%%%			Exercise ODE		%%%
%%%%%%%%%%%%%%%%%%%%%%%%%%%%%%%%%%%

\subsection*{ODE System}

\textit{Quad-well potential:
\begin{center}
\begin{align*}
	V(z) & = (\vert z_1 \vert - 1)^2 + (\vert z_2 \vert - 1)^2
\end{align*}
\end{center}
}



%%%%%%%%%%%%%%%%%%%
%%%	Homework 1)	%%%
%%%%%%%%%%%%%%%%%%%

\subsubsection*{Homework 1}

The ODEs (from what I understood):
$$
\frac{dz_1}{dt} = -\frac{\partial V}{\partial z_1}
$$

$$
\frac{dz_2}{dt} = -\frac{\partial V}{\partial z_2}
$$

$$
\frac{\partial V}{\partial z_1} = \frac{\partial}{\partial z_1} \left[ (|z_1|-1)^2 + (|z_2|-1)^2 \right] = \frac{\partial}{\partial z_1} \left( (|z_1|-1)^2 \right) = 2(|z_1|-1) \cdot \frac{\partial}{\partial z_1}(|z_1|-1)
$$

$$
\frac{\partial}{\partial z_1} |z_1| = \text{sgn}(z_1)
$$

$$
\frac{\partial V}{\partial z_1} = 2(|z_1|-1) \cdot \text{sgn}(z_1)
$$

For the $z_1$ direction:
$$
\frac{dz_1}{dt} = - \frac{\partial V}{\partial z_1} = -2(|z_1|-1) \cdot \text{sgn}(z_1)
$$
Since $\text{sgn}(z_1) \cdot |z_1| = z_1$:
$$
\frac{dz_1}{dt} = 2(1-|z_1|) \cdot \text{sgn}(z_1) = 2(\text{sgn}(z_1) - |z_1|\text{sgn}(z_1)) = 2(\text{sgn}(z_1) - z_1)
$$
Analogous for the $z_2$ direction:
$$
\frac{dz_2}{dt} = - \frac{\partial V}{\partial z_2} = 2(\text{sgn}(z_2) - z_2)
$$

\end{document}




