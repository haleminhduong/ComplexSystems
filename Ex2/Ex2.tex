\documentclass[]{scrartcl}
%\usepackage{graphicx}
%\usepackage{amsmath} % For equation alignment
\usepackage{mathtools} % For some extra math-related functionality, such as matrix*
%\usepackage{hyperref} % For using \autoref
%\usepackage{listings} % For adding code to the latex document
%\usepackage{caption} % For adding captions
\usepackage{color} % For color in code
\usepackage{nicematrix} % For creating matrices with outer rows and columns, with dsahed line separators. Documentation: https://ctan.org/pkg/nicematrix
%\usepackage{float} % For better control over float environments
%\usepackage[utf8]{inputenc} % this is needed for umlauts
%\usepackage[ngerman]{babel} % this is needed for umlauts
%\usepackage[T1]{fontenc}    % this is needed for correct output of umlauts in pdf




% Opening / Title
\title{Complex Systems in Bioinformatics \\ \vspace{2mm} Assignment 2 \\ \vspace{2mm}}
\subtitle{Lecturers: Prof. Dr. Max von Kleist, Prof. Dr. Jana Wolf, Prof. Dr. Martin Vingron}
\author{Kristian Reinhart, 4474140 \\ Duong Ha Le Minh: 5314209}
\newenvironment{tightcenter}{%
  \setlength\topsep{0pt}
  \setlength\parskip{0pt}
  \begin{center}
}{%
  \end{center}
}




%%%%%%%%%%%%%%%%%%%%%%%%%%%%%%%%%%%
%%%	Begin actual document	%%%
%%%%%%%%%%%%%%%%%%%%%%%%%%%%%%%%%%%


\begin{document}




\maketitle




\section*{2. Assignment}

\subsection*{Task 1a)}

\textit{You are given the following ODE-system [\dots].}
\\
\\
In the ODE we first identify the reaction rates $r_1$, $r_2$ and $r_3$.


\begin{center}
\noindent \begin{minipage}{.5\linewidth}
$
\begin{matrix*}[c]
	\frac{d}{dt} X_1 & = & - & \underbrace{k_1 * x_1 * x_2}	&	& 				   &   & \\ 
					 &   &   & 							r_1 &   &				   &   & \\
	\frac{d}{dt} X_2 & = & - & \underbrace{k_1 * x_1 * x_2}	& + & \underbrace{k_2} & + & \underbrace{k_3 * x_3 * x_2} \\
					 &   &   & 							r_1 &   &			   r_2 &   & 						  r_3 \\
	\frac{d}{dt} X_3 & = & - & \underbrace{k_3 * x_3 * x_2} &   &				   &   & \\
					 &   &   & 							r_3 &   &				   &   & 
\end{matrix*}
$
\end{minipage}
\end{center}

\noindent With the reaction rates and the ODE we can reconstruct the stoichiometric matrix $S$ and rate function vector $R$:

% Have the matrix in a float-environment and center it, looks nicer
\begin{center}
\noindent \begin{minipage}{.4\linewidth}
$
S =
\begin{pNiceMatrix}[first-row,last-col,nullify-dots]
	R_1	&	R_2 &	R_3 &	 \\
	 -1 &	  0 &	  0 &	X_1 \\
	 -1 &	  1 &	  1 &	X_2 \\
	  0	&	  0 &	 -1 &	X_3 \\
\end{pNiceMatrix}
$
\end{minipage}
\noindent \begin{minipage}{.4\linewidth}
$
R =
\begin{pNiceMatrix}[last-col,nullify-dots]
	k_1 * X_1 * X_2 & r_1 \\
	k_2				& r_2 \\
	k_3 * X_3 * X_2 & r_3 \\
\end{pNiceMatrix}
$
\end{minipage}
\end{center}

%\textbf{\textcolor{red}{*: Soll da 1 oder 0 hin?}}
%in dem Tutorium wurde von "Products - Educts" gesprochen, aber ich weiß nicht, ob da die rate functions $r_1$, $r_2$ und $r_3$ oder die tatsächliche Variablen $X_1$, $X_2$ und $X_3$ gemeint waren.
%\\
%In ersterem Fall stimmt die Zeile, da es $-r_1 + r_2 + r_3$ und somit $-1 | 1 | 1$ ist.
%\\
%In letzterem Fall müsste es 0 sein, da $ - X_1 - X_2 + 0 + X_3 + X_2$ und somit $-1 | 0 | 1$, was aber für mich nicht sehr viel Sinn macht.
%\\
%Ich bin relativ sicher, dass der erste Fall korrekt ist, wollte aber nochmal nachfragen.
%
%\vspace{1cm}
%\noindent\rule[0.5ex]{\linewidth}{1pt}
%\vspace{1cm}
%Ein Modell wird hier zwar nicht gefordert, aber als Übung habe ich es auch nochmal mit drauf geschrieben. Das lösche ich bevor der Zettel abgegeben wird.
%\\
%Wäre aber schön zu wissen, ob das so korrekt ist:
%
%\begin{center}
%\noindent \begin{minipage}{.4\linewidth}
%$
%\begin{matrix*}[c]
%	R_1 & : & x_1 + x_2 & \rightarrow & \emptyset \\
%	R_2 & : & \emptyset & \rightarrow & x_2 \\
%	R_3 & : & x_3		& \rightarrow & x_1
%\end{matrix*}
%$
%\end{minipage}
%\end{center}
%
%Die Frage die ich hier habe ist zu $R_2$. Da ist ja:
%\\
%$r_1 = k_1 * X_1 * X_2$
%\\
%$r_3 = k_1 * X_3 * X_2$
%\\
%In der Gleichung $\frac{d}{dt} X_2 = -k_1 * x_1 * x_2 + k_2 + k_3 * x_3 * x_2$
%\\
%\\
%Mir stellt sich dann die Frage, ob $R_2$ dann nicht irgendwie anders aussehen müsste. Oder spezifisch, wie ich von einem Modell und stoichiometrischen Matrix $S$ auf den reaction rate vector $R$ komme, das erschließt sich mir nicht so ganz. Wäre super sich da mal zusammen setzen zu können.

\end{document}
